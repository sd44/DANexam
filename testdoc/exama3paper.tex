\documentclass[a3paper, landscape, twocolumn, twoside, zihao = 5,
unicodeGBMath, % fontset=sourcesans, % sourcesans is my ctex fontset
]{ctexart}
\usepackage[lmargin]{danexam}

% \usepackage[math-style=ISO, bold-style=ISO]{unicode-math} %注意,unicode-math与被其认为过时的bm包不兼容,不要\usepackage{bm}
% % TODO:unicode-math较为方便易用和先进,并且较容易由ISO改为GB国标,详情请见
% % https://github.com/latexstudio/ChinaThesis/wiki/GB-math
% \setmathfont[Scale = 1.1]{Libertinus Math}  

\setmainfont[Scale=1.1, SlantedFont = {Libertinus Serif Italic}]{Libertinus Serif}
\setsansfont{TeX Gyre Heros}
\setmonofont[Scale=1]{Libertinus Mono}

\begin{document}

\title{\zihao{-3}  2018年火星高等学校招生统一考试\\ [-5pt]
  \zihao{-2} \heiti 文理综合}
\maketitle

{\bfseries \heiti 注意事项}:
\begin{enumerate}
\item 作弊请不要被老师发现。
\item 一旦被发现作弊,不要试图以空口求情或小恩小惠来逃避处罚。
\item “每个人都有一个他无法拒绝的价格。”
\end{enumerate}

\group{判断题:本大题共4小题,每小题5分,共20分。判断下列各题,正确的在题后括号内打\ding{52},错的打\ding{56}。}

\begin{qus}
  
\item 人类一思考,上帝就发笑。\dui
\item 尼采说道:“世上只有一个基督徒,他已经死了”,这句话中所说的“他”是指彼得吗?\cuo
\item 马克思复活后说道:“全世界无产者,原谅我!”,马克思是全盘否定了自己吗?\cuo
\item 你的人生就是一个笑话吗?\dui

\end{qus}

\group{选择题:本大题共4小题,每小题10分,共40分。在每小题给出的四个选项中,只有一项是符合题目要求的。}

\begin{qus}
\item  已知集合~$A=\{x\mid {x-1}<3 \}$,集合~$B=\{y|
  y=x^2+2x+1,x\in\mathbb{R}\}$, 则~$A\cap \complement_U B$~为\choice{C}

  \begin{tasks}(2)
    \task $[\,0,4)$ \task $(-\infty,-2\,]\cup[4,+\infty)$ \task $(-2,0)$
    \task $(0,4)$
  \end{tasks}

\item  若~$a$、$b$~是直线, $\alpha$、$\beta$~是平面,
  则以下命题中真命题是\choice{D}

  \begin{tasks}(1)
    \task 若~$a$、$b$~异面, $a\subset\alpha$,$b\subset\beta$, 且~$a\perp b$, 则~$\alpha\perp\beta$
    \task 若~$a\parallel b$, $a\subset\alpha$, $b\subset\beta$,则~$\alpha\parallel\beta$
    \task 若~$a\parallel \alpha$, $b\subset\beta$, 则~$a$、$b$ 异面
    \task 若~$a\perp b$, $a\perp\alpha$,$b\perp\beta$, 则~$\alpha\perp\beta$
  \end{tasks}

\item 某城市为了解游客人数的变化规律,提高旅游服务质量,收集并整理了2014年1
  月至2016年12月期间月接待游客量(单位:万人)的数据(图1).

  根据图1,下列结论错误的是\choice{A}

  \begin{figure}[!htb]
    \centering
    \includegraphics[width=0.8\linewidth]{zxt.jpg}
    \caption{月接待游客量折线图}
    % \label{fig:myphoto}
  \end{figure}%

  \begin{tasks}(1)
    \task 月接待游客量逐月增加 
    \task 年接待游客量逐年增加
    \task 各年的月接待游客量高峰期大致在7,8月份 
    \task 各年1月至6月的月接待游客量相对7月至12月,波动性更小,变化比较平稳 
  \end{tasks}

\item 已知双曲线 $C\colon\,\tfrac{x^2}{a^2}-\tfrac{y^2}{b^2}=1\,(a>0,b>0)$
  的一条渐近线方程为 $y=\tfrac{\sqrt{5}}{2}x$ ,且与椭
  圆$\tfrac{x^2}{12}+\tfrac{y^2}{3}=1$ 有公共焦点,则 $C$ 的方程为\choice{B}

  \begin{tasks}(4)
    \task $\tfrac{x^2}{8}-\tfrac{y^2}{10}=1$ \task $\tfrac{x^2}{4}-\tfrac{y^2}{5}=1$ \task $\tfrac{x^2}{5}-\tfrac{y^2}{4}=1$ \task $\tfrac{x^2}{4}-\tfrac{y^2}{3}=1$ 
  \end{tasks}

\end{qus}

\group{填空题:本题共4小题,每小题5分,共20分。}

\begin{qus}
  
\item 若函数~$f(x)=x^{6m^2-5m-4}\,(m\in\mathbb{Z})$~的图像关于~$y$~轴对称,
  且~$f(2)<f(6)$, 则~$f(x)$~的解析式为 \gapline{$f(x)=x^{-4}$}.

\item 若~$f(x+1)=x^2\,(x\leq0)$, 则$f^{-1}(1)=$ \gapline{$0$}.

\item 已知~$f(x)=1-\textbf{c}_8^1x+\textbf{c}_8^2x^2-\textbf{c}_8^3x^3+\cdots+\textbf{c}_8^8x^8$,
  则~$f\left(\tfrac{1}{2}+\tfrac{\sqrt{3}}{2}\textbf{i}\right)$ 的值是\gapline{$-\tfrac{1}{2}-\tfrac{\sqrt{3}}{2}\textbf{i}$}.

\item 马克思曾说“从前的一切唯物主义(包括费尔巴哈的唯物主义)的主要缺点是:
  对对象、现实、感性,只是从客体的或者直观的形式去理解,而不是把它们当作感性的人
  的活动,当作实践去理解,不是从主体方面去理解。“,所以在他看来,“哲学和对现实世界的研究这两者的关系就像\gapline{手淫}和\gapline{性爱}的关系一样。”

\end{qus}

\clearpage

\group{简答题。}

\begin{qus}
  
\item 已知复数~$z$ 满足:${z}-z^*=\dfrac{10}{1-w\textbf{i}}$(其中~$z^*$
  是~$z$ 的共轭复数).

  \begin{qus}
  \item 求复数 $z$ ;
    \begin{qus}
    \item 我只是占位,显示测试效果的,
    \item 我只是占位,显示测试效果的,
    \item 我只是占位,显示测试效果的,
    \end{qus}
  \item 若复数 $w=\cos\theta+\textbf{i}\sin\theta\,(\theta\in\mathbb{R})$, 求~${z-2}$ 的取值范围.
  \end{qus}

  \solution{
    1) $z=3+4\check{}{i}$

    2) ${z-w}\in[4,6]$
  }

  \vspace*{7cm}


\item 已知复数~$z$ 满足:${z}-z^*=\dfrac{10}{1-w\textbf{i}}$(其中~$z^*$
  是~$z$ 的共轭复数).

  \solution{}

  \vspace*{4cm}
\end{qus}

\group{作文题。}

\begin{qus}

\item 阅读下面黑格尔《历史哲学》中的一段话,思考并提出你对这段话的理解。

  \begin{quotation}
    \kaishu 经验或曰历史给我们的教训却是,人民和政府从来就没有从历史学到任
    何东西,从未依照其本应从历史中抽绎出来的教训行事。每个时代都有它特殊的处境,
    都具有一种个别的情况,使它的举动行事,不得不全由自己来考虑、自己来解决。当重
    大事件纷乘交迫的时候,一般笼统的信条毫无裨益。回忆过去的同样情形,也是徒劳无
    功的。一个灰色的回忆不能抗衡``现在''的生动和自由。
  \end{quotation}

  \drawcomposition[0.8]{20}{18}
\end{qus}

\end{document}
